\section{Theoretical Analysis}
\label{sec:analysis}

In this section, the circuit shown in Figure \ref{fig:t2} is analysed
theoretically, analysing the circuit for $t<0$, calculating the equivalent resistance, determining the natural and forced solutions and superimposing them to find the total solution.

\subsection{Envelope detector}
For t$<$0,  $v_s(t)= V_s(t)$,  it is a DC circuit. We can determine the voltges in all nodes and currents in all branches using the nodal method.
Since this is a linear circuit,  we apply Ohm's Law,  $V_i= R_i * I$ and the Kirchoff Current Law (KCL),  $\sum I_i = 0$.

We get the following equation,  in matrix form:

\begin{equation}
\label{eq:matrixeq1}
\begin{bmatrix}
    -G_1 & G_1+G_2+G_3 & -G_2 & 0 & -G_3 & 0 & 0 & 0\\
    0 & -G_2-K_b & G_2 & 0 & K_b & 0 & 0 & 0\\
    0 & K_b & 0 & 0 & -G_5-K_b & G_5 & 0 & 0\\
    0 & 0 & 0 & -G_6 & 0 & 0 & G_6+G_7 & -G_7\\
    1 & 0 & 0 & -1 & 0 & 0 & 0 & 0\\
    0 & 0 & 0 & 0 & 0 & 0 & 0 & 1\\
    0 & 0 & 0 & -K_c*G_6 & 1 & 0 & K_c*G_6 & -1\\
    0 & -G_3 & 0 & -G_4 & G_4+G_3+G_5 & -G_5 & -G_7 & G_7
\end{bmatrix}
\cdot
\begin{bmatrix}
V_1 \\
V_2 \\
V_3 \\
V_4 \\
V_5 \\
V_6 \\
V_7 \\
V_8 
\end{bmatrix}
=
\begin{bmatrix}
0 \\
0 \\
0 \\
0\\
V_s\\
0 \\
0 \\
0
\end{bmatrix}
\end{equation}


This equation solved using octave yields the following results:

\begin{table}[H]
    \centering
    \begin{tabular}{|l|r|}
      \hline    
      {\bf Variable} & {\bf Value [A or V]} \\ \hline
      %V_1 & 5.13612248730V \\ \hline 
V_2 & 4.88464690881V \\ \hline 
V_3 & 4.36195145882V \\ \hline 
V_4 & -0.00000000000V \\ \hline 
V_5 & 4.92000960269V \\ \hline 
V_6 & 5.69027079572V \\ \hline 
V_7 & -1.96654083449V \\ \hline 
V_8 & -2.94453891610V \\ \hline 
I_1 & 0.00024147744V \\ \hline 
I_2 & 0.00025321233V \\ \hline 
I_3 & -0.00001173489V \\ \hline 
I_4 & -0.00120106056V \\ \hline 
I_5 & -0.00025321233V \\ \hline 
I_6 & 0.00095958312V \\ \hline 
I_7 & 0.00095958312V \\ \hline 
I_S & -0.00024147744V \\ \hline 
I_b & -0.00025321233V \\ \hline 
I_c & -0.00000000000V \\ \hline 
I_e & -0.00095958312V \\ \hline 

    \end{tabular}
    \caption{Node Analysis Results for t$<$0}
    \label{tab:nodeanalysis}
  \end{table}
  
  
\subsection{Voltage regulator}
Now,  we have to determine the equivalent resistance $R_{eq}$ as seen from the capacitor terminals. We take out all the independent voltage sources (make $v_s=0$) and replace the capacitor with a voltage source $Vx= V(6)-V(8)$. The values of$ V(6)$ and $V(8)$ were already obtained via nodal analysis in the previous subsection. To determine the current $I_x$ supplied by $V_x$ we run mesh analysis. This is necessary because the resistors are arranged in such a way that they cannot be simplified into an equivalent resistor by applying the usual equations for resistors in series and in parallel. The mesh method gives us these equations in matriz form:


\begin{equation}\label{eq:matrixeq2}
\begin{bmatrix}
 R_1+R_3+R_4 & -R_3 & -R_4 & 0\\
    -K_bR_3 &  K_bR_3-1 & 0 & 0\\ 
    -R_4 & 0 & R_4+R_6+R_7-K_d & 0\\
    0 & -R_5 & K_d & R_5 
\end{bmatrix}
\cdot
\begin{bmatrix}
I_A\\
I_B \\
I_C \\
I_D \\

    \end{bmatrix}
=
    \begin{bmatrix}
0 \\
0 \\
0 \\
V_x \\

    \end{bmatrix}
  \end{equation}

This yields the following results:

\begin{table}[H]
    \centering
    \begin{tabular}{|l|r|}
      \hline    
      {\bf Variable} & {\bf Value [A or V]} \\ \hline
    %  V_x & 8.63480971182V \\ \hline 
I_x & 0.00283856995V \\ \hline 
R_{equiv} & 3041.95770117000V \\ \hline 

    \end{tabular}
    \caption{Equivalent resistance}
    \label{tab:equivalentresistance}
  \end{table}


For the time constant:

$\tau = R_{eq} \cdot C = 0.00313933181s$





  \end{table}
  

