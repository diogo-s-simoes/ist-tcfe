\section{Theoretical Analysis}
\label{sec:analysis}

In this section, the circuit shown in Figure \ref{fig:t1} is analysed
theoretically.

\section{Mesh analysis}
We considered 4 meshes delimited by the nodes ABDE, BCEF, DEGH and EFH and named them MA,MB,MC,MD, respectively. Since this is a linear circuit, we can apply to each one of these meshes the Kirchhoff Voltage Law (KVL):


\begin{equation}
  \sum V_i = 0
  \label{eq:KVL}
\end{equation}

Applying Ohm's Law:
\begin{equation}
  V_i= R_i * I
  \label{eq:OhmLaw}
\end{equation}

We get the following equations:
\begin{equation}
    
    V_a = (R_1+R_2+R_3)I_MA - R_3*I_MB - R_4*I_MC\\
  
    -K_b*R_3*I_MA + I_b(K_b*R_3-1)=0\\
  
    -I_MA*R_4 + I_MC(R_4-K_c+R_6+R_7)=0\\
   
    I_MD= -I_d

  \label{eq:malhas}
\end{equation}

In matrix form:
  \begin{equation}\label{eq:matrixeqn}
\begin{bmatrix}
R_1+R_2+R_3 & -R_3 & -R_4 & 0 \\

-K_b*R_3 & K_b* R_3-1 & 0 & 0 \\

-R_4 & 0 & R_4-K_c+R_6+R_7 & 0 \\

0 & 0 & 0 & 1 \\
\end{bmatrix}


    \cdot
    \begin{bmatrix}
      I_MA \\
      I_MB \\
      I_MC \\
      I_MD \\
    \end{bmatrix}
    =
    \begin{bmatrix}
      V_a \\
      0 \\
      0 \\
      -I_d\\
    \end{bmatrix}
  \end{equation}





