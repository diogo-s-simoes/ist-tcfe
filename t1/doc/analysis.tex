\section{Theoretical Analysis}
\label{sec:analysis}

In this section, the circuit shown in Figure \ref{fig:t1} is analysed
theoretically.

\subsection{Mesh analysis}
We considered 4 meshes delimited by the nodes ABDE, BCEF, DEGH and EFH and named them MA,MB,MC,MD, respectively. Since this is a linear circuit, we can apply to each one of these meshes the Kirchhoff Voltage Law (KVL):


\begin{equation}
  \sum V_i = 0
  \label{eq:KVL}
\end{equation}

Applying Ohm's Law:
\begin{equation}
  V_i= R_i * I
  \label{eq:OhmLaw}
\end{equation}

We get the following equations:
\begin{equation}
    
     V_a = (R_1+R_2+R_3)I_{MA} - R_3I_{MB} - R_4I_{MC}\\
  
    -K_bR_3I_{MA} + I_b(K_bR_3-1)=0\\
  
    -I_{MA}R_4 + I_{MC}(R_4-K_c+R_6+R_7)=0\\
   
    I_{MD}= -I_d

  \label{eq:malhas}
\end{equation}

In matrix form:
  \begin{equation}\label{eq:matrixeqn}
\begin{bmatrix}
R_1+R_2+R_3 & -R_3 & -R_4 & 0 \\

-K_b*R_3 & K_b* R_3-1 & 0 & 0 \\

-R_4 & 0 & R_4-K_c+R_6+R_7 & 0 \\

0 & 0 & 0 & 1 
\end{bmatrix}
\cdot
\begin{bmatrix}
I_{MA} \\
I_{MB} \\
I_{MC} \\
I_{MD} 
    \end{bmatrix}
=
    \begin{bmatrix}
V_a \\
0 \\
0 \\
-I_d
    \end{bmatrix}
  \end{equation}

This equation solved using octave yields the following results:

\begin{table}[H]
    \centering
    \begin{tabular}{|l|r|}
      \hline    
      {\bf Variable} & {\bf Value [A or V]} \\ \hline
      VA = 8.08066140340V
VB = 7.82918582491V
VC = 7.30649037492V
VD = 2.94453891610V
VE = 7.86454851879V
VF = 11.77414152690V
VG = 0.97799808161V
VH = 0.00000000000V
I1 = 0.00024147744A
I2 = 0.00025321233A
I3 = -0.00001173489A
I4 = -0.00120106056A
I5 = -0.00128522267A
I6 = 0.00095958312A
I7 = 0.00095958312A
IA = 0.00024147744A
IB = -0.00025321233A
IC = -0.00007242723A
ID = 0.00103201034A

    \end{tabular}
    \caption{Mesh Analysis Results}
    \label{tab:meshA}
  \end{table}

\subsection{Node analysis}
The Kirchhoff Current Law (KCL) states that,for each node, the current from every branch conected must sum to 0:

\begin{equation}
  \sum I_i = 0
  \label{eq:KCL}
\end{equation}


Using KCL and Ohm's law (\ref{eq:OhmLaw}) we can obtain an equation for each node that isn't connected to a voltage source.

\begin{equation}
    
  (V_B-V_A)G_1+(V_B-V_C)G_2+(V_B-V_E)G_3=0\\
  
  (V_C-V_B)G_2+(V_E-V_B)K_b-I_d=0\\
  
  (V_F-V_E)G_5+(V_B-V_E)K_b=0\\
  
  (V_G-V_D)G_6+(V_G-V_H)G_6=0

  \label{eq:nodos}
\end{equation}


In nodes that are conected directly to voltage sources we cannot obtain equations as the ondes above, however, by using supernodes, we can obtain pairs of equations for each pair of nodes connected to a source.
\par
The supernodes we'll be using are couplings of two nodes conected by a voltage source. The first two equations come from the fact that we know the voltage of each source, so we obtain

 \begin{equation}
  V_B-V_E=V_a\\
  
  V_E-V_H=K_c(V_D-V_G)G_6
  \label{eq:supenodosV}
\end{equation}

We can obtain another equation by fixing one node as groung.

 \begin{equation}
  V_H = 0V
  \label{eq:supenodosGND}
\end{equation}

We also know that equation (\ref{eq:KCL}) aplies to supernodes, and from this we can derive the last linearly independent equation



 \begin{equation}
  I_d+(V_H-V_G)G_7+(V_E-V_F)G_5+(V_E-V_B)G_3+(V_E-V_D)G_4=0
  \label{eq:supenodosI}
\end{equation}

All the previous equations can be compounded into a matrix
  \begin{equation}\label{eq:matrixeqn2}
\begin{bmatrix}
    -G_1 & G_1+G_2+G_3 & -G_2 & 0 & -G_3 & 0 & 0 & 0\\
    0 & -G_2-K_b & G_2 & 0 & K_b & 0 & 0 & 0\\
    0 & K_b & 0 & 0 & -G_5-K_b & G_5 & 0 & 0\\
    0 & 0 & 0 & -G_6 & 0 & 0 & G_6+G_7 & -G_7\\
    1 & 0 & 0 & -1 & 0 & 0 & 0 & 0\\
    0 & 0 & 0 & 0 & 0 & 0 & 0 & 1\\
    0 & 0 & 0 & -K_c*G_6 & 1 & 0 & K_c*G_6 & -1\\
    0 & -G_3 & 0 & -G_4 & G_4+G_3+G_5 & -G_5 & -G_7 & G_7
\end{bmatrix}
\cdot
\begin{bmatrix}
V_A \\
V_B \\
V_C \\
V_D \\
V_E \\
V_F \\
V_G \\
V_H 
    \end{bmatrix}
=
    \begin{bmatrix}
0 \\
0 \\
I_d \\
0\\
V_a \\
0 \\
0 \\
-I_d
    \end{bmatrix}
  \end{equation}

This equation solved using octave yields the following results:

  \begin{table}[H]
    \centering
    \begin{tabular}{|l|r|}
      \hline    
      {\bf Variable} & {\bf Value [A or V]} \\ \hline
      VA = 8.08066140340V
VB = 7.82918582491V
VC = 7.30649037492V
VD = 2.94453891610V
VE = 7.86454851879V
VF = 11.77414152690V
VG = 0.97799808161V
VH = 0.00000000000V
I1 = 0.00024147744A
I2 = 0.00025321233A
I3 = -0.00001173489A
I4 = -0.00120106056A
I5 = -0.00128522267A
I6 = 0.00095958312A
I7 = 0.00095958312A
IA = 0.00024147744A
IB = -0.00025321233A
IC = -0.00007242723A
ID = 0.00103201034A

    \end{tabular}
    \caption{Node Analysis Results}
    \label{tab:nodeA}
  \end{table}
