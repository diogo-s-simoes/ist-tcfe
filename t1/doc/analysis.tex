\section{Theoretical Analysis}
\label{sec:analysis}

In this section, the circuit shown in Figure \ref{fig:t1} is analysed
theoretically.

\subsection{Mesh analysis}
We considered 4 meshes delimited by the nodes ABDE, BCEF, DEGH and EFH and named them MA,MB,MC and MD respectively
(it should be noted that all all currents $I_{MA}, I_{MB}, I_{MC}, I_{MD}$ run clockwise in their respective meshes).
Since this is a linear circuit, we can apply to each one of these meshes the Kirchhoff Voltage Law (KVL):


\begin{equation}
  \sum V_i = 0
  \label{eq:KVL}
\end{equation}

Applying Ohm's Law:
\begin{equation}
  V_i= R_i * I
  \label{eq:OhmLaw}
\end{equation}

We get the following equations:
\begin{flalign}
    V_a = (R_1+R_2+R_3) I_{MA} - R_3 I_{MB} - R_4 I_{MC}\\
    -K_bR_3I_{MA} + I_b(K_bR_3-1)=0\\
    -I_{MA}R_4 + I_{MC}(R_4-K_c+R_6+R_7)=0\\
    I_{MD}= -I_d
  \label{eq:malhas}
\end{flalign}

In matrix form:
  \begin{equation}\label{eq:matrixeqn}
\begin{bmatrix}
R_1+R_2+R_3 & -R_3 & -R_4 & 0 \\
-K_b*R_3 & K_b* R_3-1 & 0 & 0 \\
-R_4 & 0 & R_4-K_c+R_6+R_7 & 0 \\
0 & 0 & 0 & 1 
\end{bmatrix}
\cdot
\begin{bmatrix}
I_{MA} \\
I_{MB} \\
I_{MC} \\
I_{MD} 
    \end{bmatrix}
=
    \begin{bmatrix}
V_a \\
0 \\
0 \\
-I_d
    \end{bmatrix}
  \end{equation}

This equation solved using octave yields the following results:

\begin{table}[H]
    \centering
    \begin{tabular}{|l|r|}
      \hline    
      {\bf Variable} & {\bf Value [A or V]} \\ \hline
      \input{../mat/meshes.tex}
    \end{tabular}
    \caption{Mesh Analysis Results}
    \label{tab:meshA}
  \end{table}

\subsection{Node analysis}
The Kirchhoff Current Law (KCL) states that,for each node, the current from every branch conected must sum to 0:

\begin{equation}
  \sum I_i = 0
  \label{eq:KCL}
\end{equation}


Using KCL and Ohm's law (\ref{eq:OhmLaw}) we can obtain an equation for each node that isn't connected to a voltage source.

\begin{align}
  (V_B-V_A)G_1+(V_B-V_C)G_2+(V_B-V_E)G_3=0\\
  (V_C-V_B)G_2+(V_E-V_B)K_b-I_d=0\\
  (V_F-V_E)G_5+(V_B-V_E)K_b=0\\
  (V_G-V_D)G_6+(V_G-V_H)G_6=0
  \label{eq:nodos}
\end{align}


In nodes that are conected directly to voltage sources we cannot obtain equations as the ondes above, however, by using supernodes, we can obtain pairs of equations for each pair of nodes connected to a source.
\par
The supernodes we'll be using are couplings of two nodes conected by a voltage source. The first two equations come from the fact that we know the voltage of each source, so we obtain

 \begin{equation}
  V_B-V_E=V_a\\
  V_E-V_H=K_c(V_D-V_G)G_6
  \label{eq:supenodosV}
\end{equation}

We can obtain another equation by fixing one node as groung.

 \begin{equation}
  V_H = 0V
  \label{eq:supenodosGND}
\end{equation}

We also know that equation (\ref{eq:KCL}) aplies to supernodes, and from this we can derive the last linearly independent equation



 \begin{equation}
  I_d+(V_H-V_G)G_7+(V_E-V_F)G_5+(V_E-V_B)G_3+(V_E-V_D)G_4=0
  \label{eq:supenodosI}
\end{equation}

All the previous equations can be compounded into a matrix
  \begin{equation}\label{eq:matrixeqn2}
\begin{bmatrix}
    -G_1 & G_1+G_2+G_3 & -G_2 & 0 & -G_3 & 0 & 0 & 0\\
    0 & -G_2-K_b & G_2 & 0 & K_b & 0 & 0 & 0\\
    0 & K_b & 0 & 0 & -G_5-K_b & G_5 & 0 & 0\\
    0 & 0 & 0 & -G_6 & 0 & 0 & G_6+G_7 & -G_7\\
    1 & 0 & 0 & -1 & 0 & 0 & 0 & 0\\
    0 & 0 & 0 & 0 & 0 & 0 & 0 & 1\\
    0 & 0 & 0 & -K_c*G_6 & 1 & 0 & K_c*G_6 & -1\\
    0 & -G_3 & 0 & -G_4 & G_4+G_3+G_5 & -G_5 & -G_7 & G_7
\end{bmatrix}
\cdot
\begin{bmatrix}
V_A \\
V_B \\
V_C \\
V_D \\
V_E \\
V_F \\
V_G \\
V_H 
    \end{bmatrix}
=
    \begin{bmatrix}
0 \\
0 \\
I_d \\
0\\
V_a \\
0 \\
0 \\
-I_d
    \end{bmatrix}
  \end{equation}

This equation solved using octave yields the following results:

  \begin{table}[H]
    \centering
    \begin{tabular}{|l|r|}
      \hline    
      {\bf Variable} & {\bf Value [A or V]} \\ \hline
      VA & 8.08066140340V \\ \hline 
VB & 7.82918582491V \\ \hline 
VC & 7.30649037492V \\ \hline 
VD & 2.94453891610V \\ \hline 
VE & 7.86454851879V \\ \hline 
VF & 11.77414152690V \\ \hline 
VG & 0.97799808161V \\ \hline 
VH & 0.00000000000V \\ \hline 
I1 & 0.00024147744A \\ \hline 
I2 & 0.00025321233A \\ \hline 
I3 & -0.00001173489A \\ \hline 
I4 & -0.00120106056A \\ \hline 
I5 & -0.00128522267A \\ \hline 
I6 & 0.00095958312A \\ \hline 
I7 & 0.00095958312A \\ \hline 
Ia & 0.00024147744A \\ \hline 
Ib & -0.00025321233A \\ \hline 
Ic & -0.00007242723A \\ \hline 
Id & 0.00103201034A \\ \hline 

    \end{tabular}
    \caption{Node Analysis Results}
    \label{tab:nodeA}
  \end{table}
