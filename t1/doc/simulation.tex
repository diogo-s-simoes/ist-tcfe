\section{Simulation Analysis}
\label{sec:simulation}

\subsection{Operating Point Analysis}

Table~\ref{tab:op} shows the simulated operating point results for the circuit
under analysis. When compared to the theoretical analysis results, we see the same values
up to 11 decimal places.

\begin{table}[h]
  \centering
  \begin{tabular}{|l|r|}
    \hline    
    {\bf Name} & {\bf Value [A or V]} \\ \hline
    @ca[i] & 0.000000e+00\\ \hline
@gb[i] & 0.000000e+00\\ \hline
@r1[i] & 0.000000e+00\\ \hline
@r2[i] & 0.000000e+00\\ \hline
@r3[i] & 0.000000e+00\\ \hline
@r4[i] & 0.000000e+00\\ \hline
@r5[i] & 0.000000e+00\\ \hline
@r6[i] & 0.000000e+00\\ \hline
@r7[i] & 0.000000e+00\\ \hline
v(1) & 0.000000e+00\\ \hline
v(2) & 0.000000e+00\\ \hline
v(3) & 0.000000e+00\\ \hline
v(5) & 0.000000e+00\\ \hline
v(6) & 0.000000e+00\\ \hline
v(8) & 0.000000e+00\\ \hline
v(71) & 0.000000e+00\\ \hline
v(72) & 0.000000e+00\\ \hline

  \end{tabular}
  \caption{Operating point. A variable preceded by @ is of type {\em current}
    and expressed in Ampere; other variables are of type {\it voltage} and expressed in
    Volt.}
  \label{tab:op}
\end{table}

It should be noted that nodes G1 and G2 represent the same node G, and exist seperatly
so as to allow the measuring of current $I_c$ in ngspice, for the purpose of defining the
dependent voltage source $V_c$.

