\section{Theoretical Analysis}
\label{sec:analysis}

In this section, the circuit shown in Figure \ref{fig:t2} is analysed
theoretically, using both the mesh and node methods.

\subsection{t<0}
For t<0, vs(t)= Vs(t), it is a DC circuit. We can determine the voltges in all nodes and currents in all branches using the nodal method.
Since this is a linear circuit, we apply Ohm's Law, $V_i= R_i * I$ and the Kirchoff Current Law (KCL), $\sum I_i = 0$.

We get the following equation, in matrix form:

  \begin{equation}\label{eq:matrixeqn2}
\begin{bmatrix}
    -G_1 & G_1+G_2+G_3 & -G_2 & 0 & -G_3 & 0 & 0 & 0\\
    0 & -G_2-K_b & G_2 & 0 & K_b & 0 & 0 & 0\\
    0 & K_b & 0 & 0 & -G_5-K_b & G_5 & 0 & 0\\
    0 & 0 & 0 & -G_6 & 0 & 0 & G_6+G_7 & -G_7\\
    1 & 0 & 0 & -1 & 0 & 0 & 0 & 0\\
    0 & 0 & 0 & 0 & 0 & 0 & 0 & 1\\
    0 & 0 & 0 & -K_c*G_6 & 1 & 0 & K_c*G_6 & -1\\
    0 & -G_3 & 0 & -G_4 & G_4+G_3+G_5 & -G_5 & -G_7 & G_7
\end{bmatrix}
\cdot
\begin{bmatrix}
V_1 \\
V_2 \\
V_3 \\
V_4 \\
V_5 \\
V_6 \\
V_7 \\
V_8 
    \end{bmatrix}
=
    \begin{bmatrix}
0 \\
0 \\
0 \\
0\\
V_s \\
0 \\
0 \\
0
    \end{bmatrix}
  \end{equation}


This equation solved using octave yields the following results:

\begin{table}[H]
    \centering
    \begin{tabular}{|l|r|}
      \hline    
      {\bf Variable} & {\bf Value [A or V]} \\ \hline
      %\input{../mat/oct.tex}
    \end{tabular}
    \caption{Node Analysis Results for t<0}
    \label{tab:nodeanalysist<0}
  \end{table}


